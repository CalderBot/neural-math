\documentclass[12pt]{amsart}

\usepackage{amssymb,mathrsfs,mathtools}
\usepackage[margin=1in]{geometry}


\newtheorem{theorem}{Theorem}[section]
\newtheorem{lemma}[theorem]{Lemma}
\newtheorem{proposition}[theorem]{Proposition}
\newtheorem{definition}[theorem]{Definition}
\newtheorem{example}[theorem]{Example}
\newtheorem{exercise}[theorem]{Exercise}
\newtheorem{remark}[theorem]{Remark}
\numberwithin{equation}{section}

% FONTS -----------------------------------------------------------
\newcommand\Rb{{\mathbb R}} % Reals
\newcommand\Zb{{\mathbb Z}} % Integers
\newcommand\Nb{{\mathbb N}} % Natural numbers
\newcommand\Lc{{\mathcal L}}
\newcommand\Xs{{\mathscr X}}
% ARROWS--------------------------------
\newcommand{\isom}{\simeq}
\newcommand{\wt}{\widetilde}
\newcommand{\diag}{\delta}
\newcommand{\id}{\iota}
\newcommand{\op}{\mu}
\newcommand{\To}{\longrightarrow}
\newcommand{\mTo}{\longmapsto}
\newcommand{\pipe}{\:|\:}
\newcommand{\dashto}{\dashrightarrow}
\newcommand{\dashTo}{\longdashrightarrow}
\newcommand{\dashtof}[1]{\stackrel{#1}{\dashrightarrow}}
\newcommand{\dashTof}[1]{\stackrel{#1}{\longdashrightarrow}}
\newcommand{\onto}{\twoheadrightarrow}
\newcommand{\into}{\hookrightarrow}
\newcommand{\tof}[1]{\stackrel{#1}{\rightarrow}}
\newcommand{\ltof}[1]{\stackrel{#1}{\leftarrow}}
\newcommand{\intof}[1]{\stackrel{#1}{\hookrightarrow}}
\newcommand{\Tof}[1]{\stackrel{#1}{\!\longrightarrow\!}}
\newcommand{\TOf}[1]{\stackrel{#1}{\Longrightarrow}}

% MATH-------------------------------------------
\DeclareMathOperator{\Sets}{Sets}
%\DeclareMathOperator{\lim}{lim}
% DOCUMENT -------------------------------------------------------------

\begin{document}

\title{Calculus for Deep Learners}
\maketitle


\subsection{Sets}\label{S:Sets}
Sets are possibly infinite collections of non-duplicated elements, denoted $\{a,b,c,...\}$.  The sets we will primarily encounter are:
\begin{itemize}
 \item $\phi := \{\:\}$, the \textbf{empty set}, which has no elements at all.
 \item $\Nb := \{\:0,1,2,...\}$, the \textbf{natural numbers}.
 \item $\Zb := \{\:0,\pm 1, \pm 2,...\}$, the \textbf{integers}.
 \item $\Rb$, the \textbf{real numbers}.
\end{itemize}
Let $X$ and $Y$ be sets.  $X$ is called a \textbf{subset} of $Y$ (written $X\subset Y$) when every element of $X$ is also an element of $Y$.  For example 
$$\phi\subset\Nb\subset\Zb\subset\Rb$$ 
is a chain of subsets.  Note that with this notion, $X\subset X$ for any $X$.

In practice subsets are often defined as the elements in a set satisfying some conditions:
$$ \{x\in X \pipe \text{x satisfies the conditions } \} \subset X. $$
Some examples we will encounter are:
\begin{itemize}
 \item $\Rb^{>0} \coloneqq \{ x\in \Rb \pipe x>0 \}$, the \textbf{positive real numbers}.
 \item $\Rb^{\geq 0} := \{ x\in \Rb \pipe x\geq 0 \}$, the \textbf{non-negative real numbers}.
 \item $\Rb^{\times} := \{ x\in \Rb \pipe x\neq 0 \}$, the \textbf{non-zero real numbers}.
\end{itemize}
The \textbf{set operations} we will use are:
\begin{itemize}
 \item $X \cup Y = \{ a \pipe a\in X \text{ or } a\in Y\}$, the \textbf{union} of $X$ and $Y$.
 \item $X \cap Y = \{ a \pipe a\in X \text{ and } a\in Y\}$, the \textbf{intersection} of $X$ and $Y$.
 \item $X \backslash Y = \{ a \pipe a\in X \text{ and } a\notin Y\}$, the \textbf{difference} of $X$ and $Y$.
 \item $X \times Y = \{ (x,y) \pipe x\in X \text{ and } y\in Y\}$, the \textbf{Cartesian product} of $X$ and $Y$.
\end{itemize}
The above operations can be iterated multiple times.  For example if $n$ is a fixed positive integer and $X$ is any set, we denote the n-fold cartesian product
  $$X^n := X\times \cdots \times X = \{(x_1,\dots,x_n)\pipe x_i\in X \text{ for } i\in \{1,\dots,n\}\}$$
 As a default we will use the letter $i$ for an index, so that for example $x_i$ denotes the $i$-th element of an \textbf{n-tuple} $(x_1,\dots,x_n)\in X^n$.


\subsection{Functions} Let $X$ and $Y$ be sets.  A \textbf{function} $f$ from X to Y (also called a $Y$-valued function on $X$, and also called a function with \textbf{domain} $X$ and \textbf{target} $Y$), is a rule assigning an element $f(x)\in Y$ to each element $x\in X$.  To make the domain and target clear, mathematicians often include them in the notation for a function, writing
 \[f:X\To Y\quad\text{or equivalently}\quad X\Tof{f}Y.\]
 Then \textbf{Sets}$(X,Y)$ denotes the set of all functions from $X$ to $Y$.
 
It is sometimes useful to talk about the set of values $f$ takes on a subset $U\subset X$.  This is referred to as the \textbf{image of U under f} and written $$f(U):=\{ f(x)\pipe x\in U\}\subset Y.$$
Similarly, for a subset $V\subset Y$, one can speak of the \textbf{inverse image under f of V}, which is the set $$f^{-1}(V):=\{ x\in X\pipe f(x)\in V \}$$  
Note that inverse images exist regardless of whether $f$ is an invertible function.  In fact, by definition, $f$ is \textbf{invertible} if and on ly if for every $y\in Y$, $f^{-1}(\{y\})$ consists of exactly one point.
 %It is also common to describe $f$ in terms of its values on elements (or \textbf{points}) of $X$, using the notation $X\ni x\mTo{f(x)}\in Y$.

Some fundamental classes of functions are the identity functions, the diagonal functions, and the operations:
\begin{enumerate}
\item \textbf{Identity functions:} For any set $X$, the identity function $\id$ (or $\id_X$ if more precision is needed), given by:
 \[   X\Tof{\id}X \qquad x\mTo x   \]
 \item \textbf{Diagonal functions:} For any set $X$ and $n\geq 2$, the diagonal function $\diag$ (or $\diag_X^n$ if more precision is needed), given by
 \[   X\Tof{\diag}X^n \qquad x\mTo (x,\dots,x)   \]
 \item \textbf{Operations:} For any set $X$ and $n\geq 1$, the \textbf{(n-ary) operations (on X)} are the functions $\Sets(X^n,X)$.  (When $n=1$ say \textbf{unary}, when $n=2$ say \textbf{binary}.  For example on $X=\Rb$ addition and multiplication are binary operations, and negation is a unary operation.)
\end{enumerate}

Two fundamental ways to make new functions from existing ones are:
\begin{enumerate}
\item \textbf{Composition:} Given functions$X\Tof{g} Y$ and $Y\Tof{f} Z$, the composition is the function defined by first applying $g$ and then $f$.  Explicitly:
 \[f\circ g:X\To Z \qquad x\mTo f(g(x)).\]
 \item \textbf{Cartesian Product} Given functions $X\Tof{f} Y$ and $X'\Tof{g} Y'$, the Cartesian product is a function $f\times g$ defined by the rule
 \[f\times g:X\times X' \To Y\times Y' \qquad (x,x')\mTo (f(x),g(x'))\]
\end{enumerate}

Using the fundamental function classes along with composition and Cartesian product, we get two other useful ways to build fuctions:
\begin{enumerate}
  \item \textbf{Product:} If $X\Tof{f} Y$ and $X\Tof{g} Z$ are functions with the same domain, we define their \textbf{product} by $(f,g):=f\times g \circ \delta$  (we denote the product $(f,g)$ to distinguish it from the Cartesian product).  Expressed more explicitly, the product is
 \[    (f,g):X\To Y\times Z \qquad x\mTo (f(x),g(x)). \]  
\item \textbf{Pointwise Operations:} Given any set $X$ and operation $Y^n\Tof{\mu} Y$, we can define an operation (also denoted $\mu$) on the set of functins $\Sets(X,Y)$ by so-called \textbf{pointwise} application of the operation.  Explicitly, for $f_1,\dots,f_n\in \Sets(X,Y)$ 
\[\mu(f_1,\dots,f_n):X\To Y\qquad \mu(f_1,\dots,f_n)(x):=\mu(f_1(x),\dots,f_n(x))\]
\end{enumerate}
\begin{example}[Product] A function $X\Tof{f} \Rb^n$ (often called a \textbf{vector-valued function on X}) is the same thing as a product of $n$ real-valued functions.  That is, 
\[ f\equiv (f_1,\dots,f_n):X\To \Rb^n \qquad f_i\in \Sets(X,\Rb). \]
\end{example}
\begin{example}[Pointwise Addition of Functions]
Let $\Rb^2\Tof{\mu}\Rb$ be addition, then for $f,g\in\Sets(X,\Rb)$, we have pointwise additon $\op(f,g)(x):= f(x)+g(x)$.  In this case we just write $f+g$.  Similarly $fg$ denotes the pointwise multiplication of functions. 
\end{example}


\begin{remark}Pointwise operations are obtained by function composition.  Indeed, for any set $X$ and operation $\op\in\Sets(Y^n,Y)$, the resulting pointwise operation on $\Sets(X,Y)$ is:
\[ \Sets(X,Y)^n\To\Sets(X,Y)\qquad (f_1,\dots,f_n)\mTo \op\circ(f_1,\dots,f_n)\circ\diag_X^n.\] 
\end{remark}


\subsection{Some properties of $\Rb^n$} For any $n\in\Nb$, $$\Rb^n = \{(a_1,\dots,a_n\pipe a_i\in\Rb)$$ is the set of $n-$tuples of real numbers.  But $\Rb^n$ admits much more structure than just that of a set.  Here is some of the structure we will implicitly equip $\Rb^n$ with:
\begin{enumerate}
\item It is a vector space, and whenever convenient we switch back and forth between viewing elements of $\Rb^n$ as vectors and as points.
\item For every $p\in [1,\infty]$ there is a measure of \textbf{magnitude (or length)} on $\Rb^n$, given by the formula
$$ \Rb^n\ni a=(a_1,\dots,a_n)\mTo |a|_p:=(|a_1|^p+\cdots+|a_n|^p)^{1/p}\in \Rb^{\geq 0}.$$
These functions are called the $p$-norms on $\Rb^n$.  The limiting case $p=\infty$ is also useful, and is given by $|a|_\infty:=max\{|a_1|,\dots,|a_n|\}$.
\item Each $p$-norm gives rise to a notion of distance $d_p$ between points by the formula:
$$d_p(v,w):=|v-w|_p.$$
Note then that $|v|_p$ is the distance from $0\in\Rb^n$ to $v$.
The case $p=2$ is called \textbf{Euclidian distance}, and agrees with the notion of distance in our day to day life. When we say or write a norm without the ``$p$'', it is implicit that $p=2$.
\end{enumerate}

\begin{exercise} Note that in $\Rb^2$, Euclidian distance as defined above is the same as the usual notion of distance between points given by the Pythagorean theorem.  Now make a drawing and use the Pythagorean theorem (twice) to show that in $\Rb^3$, Euclidian distance also agrees with the distance between points in our universe.
\end{exercise}

\subsection{Metric Spaces} A \textbf{metric (or distance function)} on a set $X$ is a function
$$d:X\times X\To \Rb^{\geq0}$$ satisfying:
 \begin{itemize}
  \item $d(x,y)=0\iff x=y$ \textbf{(definiteness)}
  \item $d(x,y) = d(y,x)$ \textbf{(symmetry)}
  \item $d(x,z)\leq d(x,y)+d(y,z)$ \textbf{(triangle inequality)}
 \end{itemize}
A \textbf{metric space} is a set equipped with a metric.

\subsection{Lmiits and Continuity} To avoid the subject of topology, and because we are interested in doing calculus, we will only define the notion of continuity for functions from subsets of $\Rb^k$ to subsets of $\Rb^n$ (for any $k,n\geq 1$).  There are several ways to define continuity, all of which give the same result.  We will phrase continuity in terms of balls in $\Rb^n$.

 For any $a\in\Rb^n$ and any $r\in \Rb^{> 0}$, the \textbf{ball (of radius r) around a} is defined as the set
 \[ B_r(a):=\{x\in \Rb^n\pipe d(x,a)<r\} \]
 Remember that $d(x,a)=|x-a|$ is the Euclidian distance.  The \textbf{punctured ball (of radius r) around a} is defined as
 \[ B_r(a)^\times:= B_r(a)\backslash\{a\}, \]
 that is, it is the ball with the point $a$ removed.
 
 Whenever $X\subset\Rb^n$ and $a\in X$, we can speak of balls in $X$, they are defined as $B_r^X(a):= B_r(a)\cap X$. 
 
\begin{exercise}
Draw $B_2(1)\subset \Rb^1$, $B_2((1,0))\subset \Rb^2$, and $B_2((1,0,0))\subset \Rb^3$.
\end{exercise}

Fix $X\subset \Rb^n$, $Y\subset \Rb^k$, $f:X\To Y$ and $x\in X$. The \textbf{limit of f at x}, written $\lim_x f$, is a point $y\in Y$ such that:
\begin{equation}
\text{ For all } r>0, \text{there exists an } R>0 \text{ such that } f(B_R^\times(x))\subset B_r(y).
\end{equation}
Such a point $y$ may not exist, but if it does exist then it is unique.  When the limit exists and equals $y$ we write $\lim_x f = y$.

 
\begin{exercise}[Practice with limits] $ $
\begin{enumerate}
\item Show that in the 1-dimensional case this definition reduces to the ``$\in -\delta$'' definition of limits for functions $f:\Rb\To\Rb$.
\item Show that $\lim_0 f$ does not exist for the function \[ \Rb\Tof{f}\Rb   \qquad  x\mTo\begin{cases}0 & x<0\\ 1 & x\geq 0\end{cases}  \]
\item Find the limit at $(0,0)$ for the function  \[ \Rb^2\Tof{f}\Rb^2   \qquad  x\mTo\begin{cases}0 & x \neq (0,0)\\ 1 & x= (0,0)\end{cases}  \]
\item Show that limits are unique if they exist.  That is, show that for any $X\tof{f}Y$, if  $\lim_x f = y$ and  $\lim_x f = y'$ then $y=y'$.
\end{enumerate}
\end{exercise}

A function $X\tof{f}Y$ is called \textbf{continuous at x} for $x\in X$ when $\lim_x f = f(x)$.  That is, not only does the limit exist but it is equal to the value of $f$ at $x$.  We say $f$ is \textbf{continuous} if it is continuous for every $x\in X$.  
\begin{proposition}[Composition of continuous functions] Let $X\Tof{f}Y$ and $Y\Tof{g} Z$ be composable functions, then  
\begin{enumerate}
\item If $f$ is continuous at $x\in X$ and $g$ is continuous at $f(x)\in Y$, then $g\circ f$ is continuous at $x$.
\item If $f$ is continuous and $g$ is continuous at every point of $f(X)$, then $g\circ f$ is continuous.
\end{enumerate}
\end{proposition}

\end{document}
